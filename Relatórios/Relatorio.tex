\documentclass[a4paper,12pt]{report}
\usepackage[utf8]{inputenc}
\usepackage[portuguese]{babel}
\usepackage{graphicx}
\usepackage{hyperref}
\usepackage{csquotes}
\usepackage{amsmath}
\usepackage{titlesec}
\usepackage{tabularx}
\usepackage{float}
\usepackage[backend=biber]{biblatex}

\addbibresource{Recursos/referencias.bib}

\setlength{\parindent}{1.5cm}
\setlength{\parskip}{0.5cm}
\renewcommand{\baselinestretch}{1.5}

\titleformat{\chapter}[display]
  {\normalfont\LARGE\bfseries}{}{0pt}{\Huge}
\renewcommand{\contentsname}{Índice}

\begin{document}

\begin{titlepage}
	\centering
	\includegraphics[width=0.5\textwidth]{Recursos/Logos/LOGO_IPB.png}\par
	\vspace{1cm}
	\textbf{Escola Superior de Tecnologia e Gestão}\\
	\textbf{Licenciatura em Engenharia Informática}

	\textbf{Desenvolvimento de Aplicações Web}

	\textbf{Projeto – Fase de Implementação}

	\textbf{My Anime Collection}

	\textbf{Martinho José Novo Caeiro}\\
	\textbf{Paulo António Tavares Abade}

	\includegraphics[width=0.5\textwidth]{Recursos/Logos/IPBejaESTIG.jpg}\par
	\vfill
	\textbf{Beja, janeiro de 2025}
\end{titlepage}

\newpage
\begin{titlepage}
	\centering
	\textbf{Instituto Politécnico de Beja}\\
	\textbf{Escola Superior de Tecnologia e Gestão}\\
	\textbf{Licenciatura em Engenharia Informática}

	\textbf{Desenvolvimento de Aplicações Web}

	\textbf{Projeto – Fase de Implementação}

	\textbf{My Anime Collection}

	\textbf{Martinho José Novo Caeiro}\\
	\textbf{Paulo António Tavares Abade}

	\textbf{DOCENTE}\\
	\textbf{Professor Luís Carlos Bruno}

	Trabalho realizado no âmbito da unidade curricular Desenvolvimento de Aplicações Web \cite{daw}

	\vfill
	\textbf{Beja, janeiro de 2025}
\end{titlepage}

\newpage
\tableofcontents
\newpage

\chapter{Introdução}
Neste relatório iremos fazer a Implementação de uma aplicação de gestão de media, que tem como objetivo ajudar a organizar o conteúdo
dos fãs de anime, ao ser possível categorizar os animes entre: Por visualizar, a visualizar, visualizados e também ser possível dar uma
classificação aos mesmos ou às listas de outros utilizadores. Irá ser feita uma API que irá conter todos os animes e os seus dados
relacionados, que será acompanhada de uma base de dados que irá guardar as informações dos utilizadores. Neste relatório será dividido em
três grandes partes, sendo estas a Análise e Desenho do Sistema cujas quais já foram desenvolvidas no relatório anterior, seguida da parte
de Implementação onde serão referidos os metodos de construção utilizados para fazer os casos de usos referidos na Analise e Desenho.

%-----------------------------------------------------------------------------------------------------------------------------%
\chapter{Análise do Sistema}
Para analisar o sistema, é necessário caracterizar os atores que realizarão as tarefas neste sistema, e quais serão as tarefas que serão
realizadas pelos mesmos, para isso serão utilizados os Casos de Uso, considerando a notação UML.

\section{Caracterização dos Atores}
Neste sistema o mesmo utilizador terá a posição de dois atores diferentes, sendo estes:
•	Visualizador
•	Gestor de Listas
Inicialmente, todos os utilizadores assumem o papel de “Visualizador” e depois irão assumir temporariamente o papel de gestor de listas.
Para explicar como funciona, será feito o uso de Personas, que são personagens fictícios para representar o futuro utilizador deste sistema.

\subsection{Persona Nº1 - Miguel Silva}
O Miguel tem 18 anos e entrou no MyAnimeCollection para descobrir novos animes. Ao visualizar o anime “One Piece” e ter-se interessado pelo
mesmo, adicionou este anime à sua lista “A Visualizar”.

\subsection{Persona Nº2 - Catarina Silvestre}
A Catarina tem 19 anos e descobriu o MyAnimeCollection pelo Miguel, e então quis verificar o perfil dele para ver que animes ele tem nas suas
listas. Ao aceder o perfil do Miguel, viu as listas que ele tinha disponíveis no seu perfil. Após ver a lista de “A Visualizar” do Miguel,
avaliou-a com 5 estrelas.

%-----------------------------------------------------------------------------------------------------------------------------%
\section{Diagrama de Casos de Uso}
Para criar o diagrama de casos de uso, foi utilizado a plataforma draw.io, respeitando a notação de casos de uso do UML.

\includegraphics[width=1\textwidth]{Recursos/Imagens/DiagramaCasoUso.png}\par

\subsection{Caso de Uso Nº1 - Consultar Animes}
\begin{table}[H]
	\centering
	\begin{tabularx}{\textwidth}{|>{\raggedright\arraybackslash}X|>{\raggedright\arraybackslash}X|>{\raggedright\arraybackslash}X|>{\raggedright\arraybackslash}X|>{\raggedright\arraybackslash}X|>{\raggedright\arraybackslash}X|}
		\hline
		\textbf{Descrição}                                                                                   & \textbf{Atores} & \textbf{Pré-Condições}                                                     & \textbf{Pós-Condições}                                       & \textbf{Cenário Principal}                                                                & \textbf{Situação de Falha}                                                            \\ \hline
		Permite ao utilizador pesquisar informações sobre um anime específico, como sinopse, episódios, etc. & Visualizador    & O utilizador deve estar conectado à internet e estar com a sessão iniciada & O anime é exibido com detalhes como sinopse, episódios, etc. & O Visualizador pesquisa um anime pelo nome, e o sistema retorna as informações detalhadas & Mostra um erro que identifica se o anime não foi encontrado ou foi um erro de conexão \\ \hline
	\end{tabularx}
\end{table}

\subsection{Caso de Uso Nº2 - Adicionar Animes a uma Lista}
\begin{table}[H]
	\centering
	\begin{tabularx}{\textwidth}{|>{\raggedright\arraybackslash}X|>{\raggedright\arraybackslash}X|>{\raggedright\arraybackslash}X|>{\raggedright\arraybackslash}X|>{\raggedright\arraybackslash}X|>{\raggedright\arraybackslash}X|}
		\hline
		\textbf{Descrição}                                                                                                  & \textbf{Atores} & \textbf{Pré-Condições}                                                     & \textbf{Pós-Condições}                   & \textbf{Cenário Principal}                                                                                                                                                         & \textbf{Situação de Falha}                                                                  \\ \hline
		Permite ao utilizador adicionar um anime à sua lista pessoal de animes para organizar seu progresso de visualização & Gestor de Lista & O utilizador deve estar conectado à internet e estar com a sessão iniciada & O anime é adicionado à lista com sucesso & O Gestor de Lista navega até o anime desejado e seleciona a opção "Adicionar à lista", categorizando-o (Por Visualizar, A Visualizar, Visualizados, ou uma criada pelo utilizador) & Mostra um erro que identifica uma falha na adição à lista por erro de sistema ou da conexão \\ \hline
	\end{tabularx}
\end{table}

\subsection{Caso de Uso Nº3 - Consultar o Perfil de outro Utilizador}
\begin{table}[H]
	\centering
	\begin{tabularx}{\textwidth}{|>{\raggedright\arraybackslash}X|>{\raggedright\arraybackslash}X|>{\raggedright\arraybackslash}X|>{\raggedright\arraybackslash}X|>{\raggedright\arraybackslash}X|>{\raggedright\arraybackslash}X|}
		\hline
		\textbf{Descrição}                                          & \textbf{Atores} & \textbf{Pré-Condições}                           & \textbf{Pós-Condições}                         & \textbf{Cenário Principal}                                                                                                                          & \textbf{Situação de Falha}                                                                      \\ \hline
		Permite visualizar o perfil e listas de outros utilizadores & Visualizador    & O perfil do outro utilizador deve ser acessível. & As informações do perfil e listas são exibidas & O Visualizador busca pelo nome de outro utilizador e, se o perfil for acessível, o sistema exibe as informações do perfil, como listas e avaliações & Mostra um erro que identifica se a falha foi por inexistência do utilizador ou falha de conexão \\ \hline
	\end{tabularx}
\end{table}

\subsection{Caso de Uso Nº4 - Avaliar uma Lista de Animes de outro Utilizador}
\begin{table}[H]
	\centering
	\begin{tabularx}{\textwidth}{|>{\raggedright\arraybackslash}X|>{\raggedright\arraybackslash}X|>{\raggedright\arraybackslash}X|>{\raggedright\arraybackslash}X|>{\raggedright\arraybackslash}X|>{\raggedright\arraybackslash}X|}
		\hline
		\textbf{Descrição}                                                       & \textbf{Atores} & \textbf{Pré-Condições}                                                                                           & \textbf{Pós-Condições}            & \textbf{Cenário Principal}                                                                                 & \textbf{Situação de Falha}                                                                           \\ \hline
		Permite ao utilizador avaliar as listas de animes de outros utilizadores & Gestor de Lista & O utilizador deve estar conectado e estar com a sessão iniciada e a lista de outro utilizador deve ser acessível & A avaliação é submetida e exibida & O utilizador acessa a lista de outro utilizador, avalia e submete a avaliação, que é registada com sucesso & Mostra um erro que identifica uma falha ao enviar a avaliação devido a erro no sistema ou na conexão \\ \hline
	\end{tabularx}
\end{table}

\subsection{Caso de Uso Nº5 (Partilhado) - Visualizar os animes com melhor avaliação}
\begin{table}[H]
	\centering
	\begin{tabularx}{\textwidth}{|>{\raggedright\arraybackslash}X|>{\raggedright\arraybackslash}X|>{\raggedright\arraybackslash}X|>{\raggedright\arraybackslash}X|>{\raggedright\arraybackslash}X|>{\raggedright\arraybackslash}X|}
		\hline
		\textbf{Descrição}                                                                                 & \textbf{Atores}                                      & \textbf{Pré-Condições}                                          & \textbf{Pós-Condições}                        & \textbf{Cenário Principal}                                                                         & \textbf{Situação de Falha}                       \\ \hline
		Exibe uma lista dos animes mais bem avaliados com base nas classificações de todos os utilizadores & Visualizador (não precisa estar com sessão iniciada) & Deve haver dados suficientes de avaliações para gerar uma lista & Lista dos animes mais bem avaliados é exibida & O Visualizador acessa a página de rankings, e o sistema exibe os animes com as melhores avaliações & Mostra um erro que identifica um erro de sistema \\ \hline
	\end{tabularx}
\end{table}

%-----------------------------------------------------------------------------------------------------------------------------%
\chapter{Desenho do Sistema}

\section{Modelação da Base de Dados}
Nesta parte será mostrada como será o sistema, considerando as suas tabelas e relações entre elas, e depois será mostrado o modelo físico
que indica as relações entre as tabelas, os atributos e o tipo dos mesmos.

\subsection{Diagrama E/R}
Este é o diagrama de entidade-relação utilizado pelo sistema do MyAnimeCollection:

\includegraphics[width=1\textwidth]{Recursos/Imagens/DiagramaER.png}\par

\subsection{Modelo Relacional}
\begin{itemize}
	\item ListaUtilizador (id\_listaUtilizador, id\_utilizador, id\_lista\_animes, nome, descricao)
	\item ListaAnimes (id\_listaAnime, id\_anime)
	\item Avaliação\_Lista\_Utilizador (id\_avaliacaoLista, id\_utilizador, id\_listaUtilizador, avaliacao)
	\item Avaliação\_Anime (id\_avaliacaoAnime, id\_utilizador, id\_anime, avaliacao)
	\item Utilizadores (id\_utilizador, nome, idade, email, password, biografia)
	\item Anime (id\_anime, sinopse, id\_categoria, n\_episodios, n\_temporadas)
	\item Categoria (id\_categoria, nome, descricao)
\end{itemize}

\subsection{Modelo Físico}
\includegraphics[width=1\textwidth]{Recursos/Imagens/ModeloFisico.png}\par

%-----------------------------------------------------------------------------------------------------------------------------%
\section{Modelação de Interfaces Gráficas com o Utilizador}

\subsection{Storyboard(s)}
Neste Storyboard podemos visualizar as movimentações entre páginas principais. Existem mais opções que podem ser feitas nas páginas,
porém serão apenas exploradas posteriormente.

\includegraphics[width=1\textwidth]{Recursos/Imagens/Storyboard.png}\par

\subsection{Interfaces do Caso de Uso 1}
\includegraphics[width=1\textwidth]{Recursos/Imagens/InterfaceCA1e2.png}\par

\subsection{Interfaces do Caso de Uso 2}
\includegraphics[width=1\textwidth]{Recursos/Imagens/InterfaceCA1e2.png}\par

\subsection{Interfaces do Caso de Uso 3}
\includegraphics[width=1\textwidth]{Recursos/Imagens/InterfaceCA3.png}\par

\subsection{Interfaces do Caso de Uso 4}
\includegraphics[width=1\textwidth]{Recursos/Imagens/InterfaceCA4.png}\par

\subsection{Interfaces do Caso de Uso 5}
\includegraphics[width=1\textwidth]{Recursos/Imagens/InterfaceCA5.png}\par

\subsection{Protótipo de Média Fidelidade}
\includegraphics[width=1\textwidth]{Recursos/Imagens/MediaFidelidade1.png}\par

\includegraphics[width=1\textwidth]{Recursos/Imagens/MediaFidelidade2.png}\par

% Melhorias efetuadas
\section{Melhorias efetuadas na Análise e Desenho do Sistema}
O Caso de Uso 5 foi aprofundado, agora sendo possivel visualizar animes por uma data especifica em vez de um periodo predefinido.

%-----------------------------------------------------------------------------------------------------------------------------%
\chapter{Implementação}
Neste capítulo são descritas as principais decisões e ações desenvolvidas na parte da implementação técnica deste projeto.

\section{Arquitetura do Sistema}
Descrever através de um diagrama de implantação (Deployment Diagram) em UML que demonstre quais são os blocos de software e hardware que compõem o sistema a desenvolver (API, Frontend/App MVC, base de dados, entre outros elementos relevantes implementados).

\section{Tecnologias Usadas}
Enumerar as tecnologias implementadas de suporte ao desenvolvimento. Associar referências bibliográficas eletrónicas a cada tecnologia.

\section{Desenvolvimento da API}
Descrever a estrutura da API REST e as principais decisões e casos de implementação.

\subsection{Especificação da Interface}
Especificar a estrutura da API REST desenvolvida e que constitui documentação necessária para um analista ou programador perceber quais são os seus endpoints (método HTTP, URL, dados a enviar no pedido e dados da resposta).

\subsection{Decisões de Implementação}
Descrever as principais decisões globais usadas na codificação da API do sistema, justificando a sua escolha, para os casos seguintes: abordagem de desenvolvimento usada (database-first, code-first, outro), escolha das rotas, definições dos controladores, definições dos modelos e segurança do sistema.

\subsection{Principais Casos Relevantes de Programação}
Explicação dos principais e mais complexos casos de codificação, exemplificando com pequenos trechos de código que permitam a um outro programador no futuro perceber esses casos.

\section{Desenvolvimento da App Frontend/MVC}
Descrever as principais decisões e casos de implementação da App frontend/MVC.

\subsection{Decisões de Implementação}
Descrever as principais decisões globais usadas na codificação da App, justificando a escolha, para os casos seguintes: escolha das rotas, definições dos controladores, definições dos modelos e segurança do sistema. Definições dos layouts das vistas (componentes das vistas).

\subsection{Principais Casos Relevantes de Programação}
Explicação dos principais casos de codificação para os casos de uso em causa, exemplificando com pequenos trechos de código e imagens das interfaces geradas pelas vistas.

%-----------------------------------------------------------------------------------------------------------------------------%
\chapter{Conclusão e Trabalho Futuro}
Nesta etapa do projeto, obtivemos resultados que consideramos bastante satisfatórios e acreditamos que este projeto apresenta um grande
potencial para o futuro. A utilização do ASP.NET, em vez do Laravel como inicialmente planeado, proporcionou-nos maior facilidade na
implementação da API, uma vez que o ASP.NET nos permitiu criar uma API e um projeto MVC num único projeto, simplificando assim a comunicação
entre a API e o Frontend. Estamos confiantes de que a aplicação tem um maior nivel de fidelidade em relação à versão feita no semestre
anterior na Unidade Curricular de Tecnologias Web e Desenvolvimento de Aplicações Móveis, onde encontramos dificuldades significativas
devido às limitações da RestDB, que possuía um limite relativamente restrito.

\printbibliography
\addcontentsline{toc}{chapter}{Bibliografia}

\end{document}