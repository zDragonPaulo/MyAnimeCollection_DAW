\documentclass[a4paper,12pt]{report}
\usepackage[utf8]{inputenc}
\usepackage{graphicx}
\usepackage{hyperref}
\usepackage{amsmath}

% Configurações de layout
\setlength{\parindent}{1.5cm}
\setlength{\parskip}{0.5cm}
\renewcommand{\baselinestretch}{1.5}

\begin{document}

% Capa
\begin{titlepage}
	\centering
	\includegraphics[width=0.5\textwidth]{Recursos/Logos/LOGO_IPB.png}\par
	\vspace{1cm}
	\textbf{Escola Superior de Tecnologia e Gestão}\\
	\textbf{Licenciatura em Engenharia Informática}

	\textbf{Desenvolvimento de Aplicações Web}

	\textbf{Projeto – Fase de Implementação}

	\textbf{My Anime Collection}

	\textbf{Martinho José Novo Caeiro}\\
	\textbf{Paulo António Tavares Abade}

	\includegraphics[width=0.5\textwidth]{Recursos/Logos/IPBejaESTIG.jpg}\par
	\vfill
	\textbf{Beja, janeiro de 2025}
\end{titlepage}

% Página de rosto
\newpage
\begin{titlepage}
	\centering
	\textbf{Instituto Politécnico de Beja}\\
	\textbf{Escola Superior de Tecnologia e Gestão}\\
	\textbf{Licenciatura em Engenharia Informática}

	\textbf{Desenvolvimento de Aplicações Web}

	\textbf{Projeto – Fase de Implementação}

	\textbf{My Anime Collection}

	\textbf{Martinho José Novo Caeiro}\\
	\textbf{Paulo António Tavares Abade}

	\textbf{DOCENTE}\\
	\textbf{Professor Luís Carlos Bruno}

	Trabalho realizado no âmbito da unidade curricular Desenvolvimento de Aplicações Web

	\vfill
	\textbf{Beja, janeiro de 2025}
\end{titlepage}

% Índice
\newpage
\tableofcontents
\newpage

% Introdução
\chapter{Introdução}

Neste relatório iremos fazer a Implementação de uma aplicação de gestão de media, que tem como objetivo ajudar a organizar o conteúdo
dos fãs de anime, ao ser possível categorizar os animes entre: Por visualizar, a visualizar, visualizados e também ser possível dar uma
classificação aos mesmos ou às listas de outros utilizadores. Irá ser feita uma API que irá conter todos os animes e os seus dados
relacionados, que será acompanhada de uma base de dados que irá guardar as informações dos utilizadores. Neste relatório será dividido em
três grandes partes, sendo estas a Análise e Desenho do Sistema cujas quais já foram desenvolvidas no relatório anterior, seguida da parte
de Implementação onde serão referidos os metodos de construção utilizados para fazer os casos de usos referidos na Analise e Desenho.

%-----------------------------------------------------------------------------------------------------------------------------%
% Análise do Sistema
\chapter{Análise do Sistema}
Para analisar o sistema, é necessário caracterizar os atores que realizarão as tarefas neste sistema, e quais serão as tarefas que serão
realizadas pelos mesmos, para isso serão utilizados os Casos de Uso, considerando a notação UML.

\section{Caracterização dos Atores}
Neste sistema o mesmo utilizador terá a posição de dois atores diferentes, sendo estes:
•	Visualizador
•	Gestor de Listas
Inicialmente, todos os utilizadores assumem o papel de “Visualizador” e depois irão assumir temporariamente o papel de gestor de listas.
Para explicar como funciona, será feito o uso de Personas, que são personagens fictícios para representar o futuro utilizador deste sistema.

\subsection{Persona Nº1 - Miguel Silva}
O Miguel tem 18 anos e entrou no MyAnimeCollection para descobrir novos animes. Ao visualizar o anime “One Piece” e ter-se interessado pelo
mesmo, adicionou este anime à sua lista “A Visualizar”.

\subsection{Persona Nº2 - Catarina Silvestre}
A Catarina tem 19 anos e descobriu o MyAnimeCollection pelo Miguel, e então quis verificar o perfil dele para ver que animes ele tem nas suas
listas. Ao aceder o perfil do Miguel, viu as listas que ele tinha disponíveis no seu perfil. Após ver a lista de “A Visualizar” do Miguel,
avaliou-a com 5 estrelas.

%-----------------------------------------------------------------------------------------------------------------------------%
\section{Diagrama de Casos de Uso}
Para criar o diagrama de casos de uso, foi utilizado a plataforma draw.io, respeitando a notação de casos de uso do UML

\subsection{Caso de Uso Nº1 - Consultar Animes}

\subsection{Caso de Uso Nº2 - Adicionar Animes a uma Lista}

\subsection{Caso de Uso Nº3 - Consultar o Perfil de outro Utilizador}

\subsection{Caso de Uso Nº4 - Avaliar uma Lista de Animes de outro Utilizador}

\subsection{Caso de Uso Nº5 - Visualizar os animes com melhor avaliação}

%-----------------------------------------------------------------------------------------------------------------------------%
% Desenho do Sistema
\chapter{Desenho do Sistema}
Em seguida são descritos os modelos propostos de interfaces gráficas com o utilizador e do modelo de dados.

\section{Modelação da Base de Dados}
Descrição dos modelos E/R e relacional/físico da base de dados. O modelo E/R deve conter todas as entidades de dados e respetivas relações que descrevem conceptualmente o sistema. O diagrama relacional/físico descreve a estrutura normalizada de tabelas e suas relações com a identificação dos seus atributos e respetivos tipos.

\subsection{Diagrama E/R}
<<...>>

\subsection{Modelo Físico}
<<...>>

\section{Modelação de Interfaces Gráficas com o Utilizador}
Em seguida são descritos os modelos das interfaces gráficas desenvolvidas suportados em protótipos de baixa/média fidelidade das interfaces gráficas com o utilizador, compostos por um ou mais storyboard e os ecrãs (wireframes) da aplicação.

\subsection{Storyboard(s)}
<<...>>

\subsection{Interfaces do Caso de Uso "1"}
<<...>>

\subsection{Interfaces do Caso de Uso "N"}
<<...>>

% Melhorias efetuadas
\section{Melhorias efetuadas na Análise e Desenho do Sistema}
Descrever as principais melhorias efetuadas após a defesa desta fase do projeto.

% Implementação
\chapter{Implementação}
Neste capítulo são descritas as principais decisões e ações desenvolvidas na parte da implementação técnica deste projeto.

\section{Arquitetura do Sistema}
Descrever através de um diagrama de implantação (Deployment Diagram) em UML que demonstre quais são os blocos de software e hardware que compõem o sistema a desenvolver (API, Frontend/App MVC, base de dados, entre outros elementos relevantes implementados).

\section{Tecnologias Usadas}
Enumerar as tecnologias implementadas de suporte ao desenvolvimento. Associar referências bibliográficas eletrónicas a cada tecnologia.

\section{Desenvolvimento da API}
Descrever a estrutura da API REST e as principais decisões e casos de implementação.

\subsection{Especificação da Interface}
Especificar a estrutura da API REST desenvolvida e que constitui documentação necessária para um analista ou programador perceber quais são os seus endpoints (método HTTP, URL, dados a enviar no pedido e dados da resposta).

\subsection{Decisões de Implementação}
Descrever as principais decisões globais usadas na codificação da API do sistema, justificando a sua escolha, para os casos seguintes: abordagem de desenvolvimento usada (database-first, code-first, outro), escolha das rotas, definições dos controladores, definições dos modelos e segurança do sistema.

\subsection{Principais Casos Relevantes de Programação}
Explicação dos principais e mais complexos casos de codificação, exemplificando com pequenos trechos de código que permitam a um outro programador no futuro perceber esses casos.

\section{Desenvolvimento da App Frontend/MVC}
Descrever as principais decisões e casos de implementação da App frontend/MVC.

\subsection{Decisões de Implementação}
Descrever as principais decisões globais usadas na codificação da App, justificando a escolha, para os casos seguintes: escolha das rotas, definições dos controladores, definições dos modelos e segurança do sistema. Definições dos layouts das vistas (componentes das vistas).

\subsection{Principais Casos Relevantes de Programação}
Explicação dos principais casos de codificação para os casos de uso em causa, exemplificando com pequenos trechos de código e imagens das interfaces geradas pelas vistas.

% Conclusão e Trabalho Futuro
\chapter{Conclusão e Trabalho Futuro}
Resumo dos principais resultados obtidos nas tarefas desenvolvidas nesta fase do projeto e identificação de pontos a desenvolver no futuro.

% Referências Bibliográficas
\chapter*{Referências Bibliográficas}
\addcontentsline{toc}{chapter}{Referências Bibliográficas}
Seguir normas APP para citação bibliográfica.

\end{document}
